%----------------------------------------------------------------------------------------
%	PACKAGES AND OTHER DOCUMENT CONFIGURATIONS
%----------------------------------------------------------------------------------------

\documentclass{resume} % Use the custom resume.cls style
\usepackage{hyperref}
\usepackage{xcolor}
\usepackage[left=0.75in,top=0.6in,right=0.75in,bottom=0.6in]{geometry} % Document margins

\name{Suman Shekhar} % Your name
\address{Kankarbagh \\ Patna, Bihar 800020} % Your address
% \address{123 Pleasant Lane \\ City, State 12345} % Your secondary addess (optional)
\address{(766)~$\cdot$~725~$\cdot$~1869 \\ sumanshekhar170219@gmail.com} % Your phone number and email

\begin{document}

%----------------------------------------------------------------------------------------
%	EDUCATION SECTION
%----------------------------------------------------------------------------------------
\begin{rSection}{Research Interest}
Physical Oceanography, Geophysical Fluid Dynamics, Computational Fluid Dynamics, High Performance Computing, Numerical Modeling, Scientific Programming, Spectral Methods, Large Eddy Simulation, Scalable Data Analysis, Open source software development

\end{rSection}

\begin{rSection}{Education}

{\bf Indian Institute of Technology (Indian School of Mines) Dhanbad} \hfill {\em Expected May 2022} \\ 
BTech in Mechanical Engineering \\
Overall GPA: 6.76/10
\end{rSection}

%----------------------------------------------------------------------------------------
%	WORK EXPERIENCE SECTION
%----------------------------------------------------------------------------------------

\begin{rSection}{Research Experience}

\begin{rSubsection}{Melbourne University}{May 2021 - Present}{Summer Internship}{Melbourne, Australia}
\item Investigating the role of double-diffusion in the formation of Antarctic offshore polynya.
\item Developed a python script for Analysis of Argo float data which contain float hydro-graphic observations from Maud rise from 2011-2018. 
\item Plots with potential temperature, salinity, Buoyancy frequency were plotted for better visualization of Mixed layer depths and convection period.
\item Large Eddy Simulation Modelling of open ocean convection with the initial condition extracted from Argo float data and Heat flux boundary conditions that were obtained by mooring data. The aim is to reproduce the event investigating the role of double-diffusion in the event. 
\end{rSubsection}
%------------------------------------------------
\begin{rSubsection}{Melbourne University}{November 2020 - February 2021}{Winter Internship}{Melbourne, Australia}
\item Modelling Open ocean convection using open source solver Oceananigans.
\item Performed Grid analysis to find optimized grid for the simulation. 
\item Developed Large Eddy Simulation Julia script that uses GPU for computation and analysed the Mixed Layer Depth results with serial code. 
\item Developed Python script that uses lazy computation using Dask(parallel computing package) to compute simulation data contained in NetCDF file.
 
\end{rSubsection}
%------------------------------------------------

\begin{rSubsection}{Universiti Tecknologi Petronas}{June 2020 - August 2020}{Research Internship}{Malasiya}
\item Performed Species transport and chemical reaction simulation of in a combustion chamber using ANSYS software. 
\item Developed a Python and MATLAB script which solves conservation equations describing convection, diffusion, diffusion and reaction sources for each component species in the fuel chamber.
\end{rSubsection}

%------------------------------------------------

\begin{rSubsection}{National University of Singapore}{May 2020 - June 2020}{Research Internship}{Singapore}
\item Learnt about Direct Numerical Simulation and FORTRAN programming language.
\item Conducted a Numerical study on Stookian flow. 
\end{rSubsection}
 %---------------------------------------------------
\begin{rSubsection}{University of Nicosia}{April 2020 - May 2020}{Research Internship}{Nicosia, Cyprus}
\item Developed a python script to simulate a virus transmission in population with varying wind speed. 
\item Analysed SIR virus transmission model. And pointed out that those model must incorporate fluid dynamics parameter that can govern virus transmission. Came up with humidity and wind speed to incorporate as a beta parameter in SIR model.  

\end{rSubsection}

\end{rSection}

%----------------------------------------------------------------------------------------
%	TEACHING EXPERIENCE
%----------------------------------------------------------------------------------------

\begin{rSection}{Teaching Experience}
\begin{rSubsection}{Computational Fluid Dynamics}{fall 2021}{Department of Chemical Engineering, Indian Institute of Technology (ISM) Dhanbad}
\item Introduced the concept of Global Spectral Analysis and introduced commercial software such as ANSYS, Converge CFD. 
\end{rSubsection}

\begin{rSubsection}{Aerodynamics}{spring 2020}{Mechismu Racing Workshop, Indian Institute of Technology (ISM) Dhanbad}
\item Workshop slides can be accessed here.\href{https://github.com/Sumanshekhar17/Mechismu-Racing-Aerodynamics}{\textcolor{blue}{GitHub}}
\end{rSubsection}
\end{rSection}
%--------------------------------------------------------------------------------

\begin{rSection}{LAB EXPERIRNCE}

\begin{rSubsection}{Aero-Acoustic Lab}{February 2019 - August 2019}{Department of Mechanical Engineering, Indian Institute of Technology (ISM) Dhanbad}
\item Tests of leading edge serration on an Aerofoil, noise reduction capabilities/reverberation room sound-absorption coefficients, sound transmission loss.
\end{rSubsection}

\begin{rSubsection}{High Performance Computing Lab}{May 2021 - present}{Department of Mechanical Engineering, Indian Institute of Technology (ISM) Dhanbad}
\item Development of High Accuracy Algorithm for scientific computing.
\end{rSubsection}

\end{rSection}

\begin{rSection}{OTHER EXPERIENCES}

\begin{rSubsection}{Ocean Decade Laboratory}{}{}
\item Volunteered for crowdsorcing ideas for OpenOceanCloud project. It is a vision for a distributed, open-source cyberinfrastructure for data-intensive oceanography research. \item It is based on open, distributed cloud architecture we already being prototyped within Pangeo. 
\item Participated in open discussion that fills a critical need for the United Nation's vision of Ocean Decade by putting cost-effective, powerful computing power and ocean data access in the hands of anyone.
\end{rSubsection}

\begin{rSubsection}{Mechismu Racing(Formula Student Team)}{February 2019 - May 2021}{Mechismu Racing, Indian Institute of Technology (ISM) Dhanbad}
\item \underline{Vehicle Dynamics}- Building a Data Analytics programs to study tyre characteristic from tyre data. This resulted in 1.3s reduction in lap time. 
\item \underline{Chassis} - Optimizing design of Formula Student car chassis which resulted in reduced chassis weight from 33kg to 29kg.
\item \underline{Steering} - Building a MATLAB program for studying Steering response data and modifying geometry accordingly.
\item \underline{Operations} - Accurate Scheduling using Time-Scaled network to reduce manufacturing phase from 6 months to 4 months.
\href{https://drive.google.com/file/d/1gxg4ES_sFsu9u8YnKFnXOZkrpout1oJu/view?usp=sharing}{\textcolor{blue}{link}}

\end{rSubsection}

\begin{rSubsection}{Kartavya NGO}{December 2018 - May 2019}{Kartavya, Indian Institute of Technology (ISM) Dhanbad}
\item Tutoring mathematics to unprivileged children for free of cost. 
\end{rSubsection}

\begin{rSubsection}{AIESEC NGO}{September 2018-January 2018}{Indian Institute of Technology (ISM) Dhanbad}
\item Attended various Leadership programs focused on \underline{Climate Action} one of the Sustainable development goals.
\item Volunteered for organising TedX talk. 
\end{rSubsection}

\end{rSection}
\begin{rSection}{PRESENTATIONS/HACKATHONS}
\begin{rSubsection}{Joint Hackathon (3rd NOAA AI Workshop - Climate Informatics)}{7 September 2021 - 10 September 2021}{}
TThe goal of this hackathon is to develop a data-driven method to predict the risk of the local community to wildfire using climatic, local development, and topography data. The historical damaged and destroyed structures/buildings from wildfire incidents reports (2000–2014) are used as the proxy to quantify the risk of wildfire for local communities. Our Team ranked third in this event. \href{https://github.com/climate-informatics/hackathon-2021/blob/main/wildfire_risk/README.md}{\textcolor{blue}{GitHub}}
\end{rSubsection}

\begin{rSubsection}{OceanHackWeek}{3 August 2021 - 6 August 2021}{}
DDenisse Fierro Arcos, Shikha Singh, \underline{Suman Shekhar}(August 2021).Project Presentation in OceanHackWeek organised by \underline{University of Washington}  on CMIP6 workflows, turning big climate projection data into useful inputs for modelling and analysis.\href{https://github.com/oceanhackweek/ohw21-proj-cmip-ard}{\textcolor{blue}{GitHub}}, \href{https://youtu.be/Y5e50p39mnQ}{\textcolor{blue}{Video presentation}}  
\end{rSubsection}

\end{rSection}
%----------------------------------------------------------
% Projects
\begin{rSection}{Projects}
\begin{rSubsection}{GPU Solver for 2D Lid Driven Cavity}{2020}{}
\item Developed a GPU solver for 2D lid-driven cavity problem, using finite difference method that solves the incompressible, isothermal 2D Navier–Stokes equations on a GPU (via CUDA).\href{https://github.com/Sumanshekhar17/3rd-year-project}{\textcolor{blue}{GitHub}}
\end{rSubsection}
\begin{rSubsection}{Conjugate Heat Transfer Analysis on Graphics Card}{2019}{}
\item Performed computational fluid dynamics analysis using ANSYS software on graphics card to optimize fan speed for maximum cooling.\href{https://skill-lync.com/student-projects/conjugate-heat-transfer-analysis-on-a-graphics-card-62}{\textcolor{blue}{link}}
\end{rSubsection}

\begin{rSubsection}{Rayleigh Taylor Instability}{2019}{}
\item Performed computational fluid dynamics analysis for the objective of finding the effect of Atwood number on instability.\href{https://skill-lync.com/student-projects/rayleigh-taylor-instability-challenge-35}{\textcolor{blue}{link}}
\end{rSubsection}

\end{rSection}


%----------------------------------------------------------------------------------------
%	TECHNICAL STRENGTHS SECTION
%----------------------------------------------------------------------------------------

\begin{rSection}{Technical Skills}

\begin{tabular}{ @{} >{\bfseries}l @{\hspace{6ex}} l }
Programming Languages & Fortran, C++, Julia, matlab, Python, CUDA, MPI, git \\
Packages \& APIs & Xarray, Dask, Pandas, Xgcm, Numpy, Metpy \\
Softwares & ANSYS, Converge CFD, Solidworks, StarCCM \\
Documentation Preparation Systems &  LaTeX (document classes: article, beamer; packages: tikz)\\
Open Source Tools & Oceananigans, CMIP6 pangeo preprocessing\\
Scientific Computing skills & Global Spectral Analysis, Fourier Analysis, FEA, FDM\\
modeling skills & Large Eddy Simulations, DNS, RANS
\end{tabular}

\end{rSection}

\begin{rSection}{references}



\begin{rSubsection}{Dr. Bishakhdatta Gayen}{Research Internship Advisor}{}
DDepartment of Mechanical Engineering\\
University of Melbourne, Australia\\
bishakhdatta.gayen@unimelb.edu.au\\

\end{rSubsection}

\begin{rSubsection}{Dr. Subramanian  Narayanan}{Aero-Acoustic Lab Advisor}{}
AAssociate Professor\\
Department of Mechanical Engineering\\
Indian Institute of Technology (ISM) Dhanbad, India\\
narayan@iitism.ac.in\\

\end{rSubsection}
\begin{rSubsection}{Dr. Pawan Kumar Singh}{Fluid Mechanics Instructor}{}
AAssistant Professor\\
Department of Mechanical Engineering\\
Indian Institute of Technology (ISM) Dhanbad, India\\
pawan@iitism.ac.in\\

\end{rSubsection}
\begin{rSubsection}{Nirmal Kumar Singh}{FSAE Project Advisor}{}
AAssociate Professor\\
Department of Mechanical Engineering\\
Indian Institute of Technology (ISM) Dhanbad, India\\
nirmal@iitism.ac.in\\

\end{rSubsection}




\end{rSection}

%----------------------------------------------------------------------------------------
%	EXAMPLE SECTION
%----------------------------------------------------------------------------------------

%\begin{rSection}{Section Name}

%Section content\ldots

%\end{rSection}

%----------------------------------------------------------------------------------------


\end{document}
